%
% ********** Resumo
%

% Usa-se \chapter*, e n�o \chapter, porque este "cap�tulo" n�o deve
% ser numerado.
% Na maioria das vezes, ao inv�s dos comandos LaTeX \chapter e \chapter*,
% deve-se usar as nossas vers�es definidas no arquivo comandos.tex,
% \mychapter e \mychapterast. Isto porque os comandos LaTeX t�m um erro
% que faz com que eles sempre coloquem o n�mero da p�gina no rodap� na
% primeira p�gina do cap�tulo, mesmo que o estilo que estejamos usando
% para numera��o seja outro.

\mychapterast{Resumo}
O crescente uso de tecnologias sem fio, aliado � necessidade de se encontrar solu��es informatizadas para problemas relacionados 
ao tr�nsito, s�o fatores que motivam o desenvolvimento deste trabalho. O presente trabalho tem como finalidade projetar um 
sistema que identifica e gerencia vagas de estacionamento atrav�s da comunica��o entre sensores sem fio presentes em cada vaga. 

\mychapterast{Abstract}

The growling demand fo wireless Technologies allied with the need to find computer solutions in order to solve problems related to transit are factors 
that motivates the development of this work. This paper aims to design a system that identifies and manages parking spaces 
through the communication between wireless sensors present in each parking space.